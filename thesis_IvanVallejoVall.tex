\documentclass[12pt]{article}
% Set font to Times (similar to Times New Roman)
\usepackage[T1]{fontenc}
\usepackage{mathptmx}
% Set font size sections and subsections
\usepackage{sectsty}
\sectionfont{\fontsize{16}{15}\selectfont}
\subsectionfont{\fontsize{16}{15}\selectfont}
% Set document margins to 3cm
\usepackage[margin=3cm]{geometry}
\usepackage{cite}
\usepackage{amsmath,amssymb}
\usepackage{ctable}
\usepackage{tabularx}
\usepackage{float}
\usepackage[sc]{caption}
\usepackage{multirow}
\usepackage{microtype}
\usepackage{epstopdf}
\usepackage{verbatim}
\usepackage{verbatimbox}
\usepackage[multiple,bottom]{footmisc}
\usepackage[title,titletoc,toc]{appendix}
\usepackage{setspace}
\usepackage{rotating}
\usepackage{lscape}
%\usepackage{pdflscape}
\usepackage{subfigure}
\usepackage[para,online,flushleft]{threeparttable}
\usepackage{hyperref}
\usepackage{bm}
\usepackage{ragged2e}
\usepackage{epsf} 
\usepackage{graphicx}
\setlength{\parskip}{1em}
\setlength{\parindent}{0em}
% to allow capitalization of cross-references
\usepackage{cleveref}
\usepackage{bbold}
\usepackage[utf8]{inputenc}
\usepackage[english]{babel}
\setcounter{MaxMatrixCols}{20}
% Set footnote indent preferences
\makeatletter
\renewcommand\@makefntext[1]{\leftskip=0.5em\hskip0em\@makefnmark#1}
\makeatother
\addtolength{\footnotesep}{2mm}
% Set hyperlink looks
\hypersetup{colorlinks,urlcolor=blue,citecolor=black,linkcolor=black}
\makeatletter
\renewcommand*{\@cite@ofmt}{\bfseries\hbox}
\makeatother
\urlstyle{same}
%Para settings
\setlength\parindent{0pt}
\linespread{1.1}
% More enumaration formats
\usepackage{enumitem}
% More tabular formats
\newcommand{\specialcell}[3][c]{% 
\begin{tabular}[#1]{@{}#2@{}}#3\end{tabular}}
% Biblio settings
\usepackage{natbib}
% Captions & Sources style
\usepackage[margin={1.5cm,1.5cm},labelfont=bf, font=small,justification=raggedright,singlelinecheck=false]{caption}
\newcommand{\source}[1]{\vspace{-3pt} \caption*{ Source: {#1}} }
% Set footnote size
\renewcommand{\footnotesize}{\small}

% Title, author, date
\author{Author: Ivan Vallejo Vall \\ Tutor: Iñigo Herguera García}
\title{%
  \vspace{-0.0cm}MASTER PROJECT \\ 
  \vspace{1cm} Measuring real broadband speeds using \\ crowdsourcing data from the Internet Foundation \\ \vspace{2cm} 
  }
\date{\vspace{1cm} Data Science Program 2016/2017 \\ 
\vspace{2cm} \includegraphics[scale=0.5]{Logo_GSE_green.png}}


\begin{document}
\linespread{1.4}\selectfont
\pagenumbering{gobble}% Remove page numbers (and reset to 1)
\maketitle
\newpage
\begin{abstract}
   Here goes the abstract-text, max 150 words.
\end{abstract}
\newpage
\tableofcontents
\newpage
\listoffigures
\listoftables
\newpage
\pagenumbering{arabic}% Arabic page numbers (and reset to 1)
\section{Introduction}
\subsection{Background}
The International Telecommunication Union -- ITU, the United Nations specialized agency for information and communication technologies (ICTs)\footnote{For more information on ITU, see \href{http://www.itu.int/en/about/Pages/default.aspx}{ITU's website}} -- is carrying out a series of pilot studies under the umbrella of the project Big Data for Measuring the Information Society (\autoref{fig:a1}).\footnote{For more information on the ITU project Big Data for Measuring the Information Society, see its \href{http://www.itu.int/en/ITU-D/Statistics/Pages/bigdata/default.aspx}{website}.} 
\vspace{1cm}
\begin{figure}[H]
    \centering
        \includegraphics[width=\linewidth]{itu_projects.pdf}
        \caption{Big Data for Measuring the Information Society -- ITU pilot projects.}
        \caption*{\textbf{Source:} \cite{margus}.}
        \label{fig:a1}
\end{figure}   


The objective of the project is to show how big data from the telecommunication industry can be used to produce new ICT indicators and replace or complement existing ones with a view to measuring the development of the Information Society worldwide. 

In particular, these pilot studies aim to be a first step towards filling in the ICT data gaps in the global indicator framework agreed for the monitoring of the 2030 Agenda for Sustainable Development \citep{interagency}, and to inform private and public stakeholders on the current status of the digital divide.      

The object of this master thesis concerns the measurements on broadband speeds carried out by the Internet Foundation in Sweden (IIS)\footnote{The Internet Foundation in Sweden is an independent public-service organization which is responsible for the operation of the top-level domains '.se' and '.nu'. IIS reinvests part of the revenues obtained from the administration of these domains in activities to promote the stability of the Internet infrastructure in Sweden and research on the Internet. As part of these activities, IIS has developed the \textit{Bredbandskollen}, a software-based Internet platform to measure actual broadband speeds. For more information, see \href{https://www.iis.se/english/what-we-do/}{IIS website}} and made available to ITU in the context of the project Big Data for Measuring the Information Society.    

\subsection{Relevance}
Broadband speed measurements matter because they are a key input to several consumer, policy and regulatory decisions:

\begin{itemize}

	\item From a \textbf{consumer perspective}, broadband speed is one of the most important factors when choosing an Internet connection. For instance, in the European Union (EU) the download speed is the second most cited factor, after price, when deciding which broadband plan to choose.\footnote{On average, 41 per cent of respondents in the EU mentioned download speed as an important factor when subscribing to an Internet connection, compared with 71 per cent of respondents citing price. The figures refer to fieldwork carried out in January 2014.} However, six out of ten EU citizens do not know the maximum download speed of their broadband Internet plan and, among those that know it, a quarter of them believe that their real speed does not correspond to the one specified in their contract. Moreover, four in ten households in the EU admit having experienced difficulties accessing content at home because of speed or capacity issues \citep{eurobarometer}. It can thus be concluded that Internet speed is both an important and a controversial factor for consumers. 
	
	\item From a \textbf{regulatory perspective}, broadband speed is one of the parameters often monitored to ensure that telecommunication operators and Internet service providers (ISPs) comply with some minimum quality-of-service (QoS) requirements. For example, Spain regulates the QoS parameters of electronic communication services by means of a service order, which includes specific Internet access parameters \citep[Annex I, Part II in ][]{boe}. In a similar fashion, the Telecom Regulatory Authority of India sets that subscribers should get a minimum of 80 per cent of the speed specified in their contract, as measured from the ISP node to the user \citep{trai2006}.         
	
	\item From a \textbf{policy perspective}, broadband speeds have wide implications concerning the initiatives undertaken in the telecommunication sector. For instance, the definition of broadband is often tied to a given minimum speed, which is subject to be revised, as was the case in India in 2014 (from 256 to 512 kbit/s) \citep{trai2014}. In the United States, the Congress asked the Federal Communications Commission (FCC) to evaluate the deployment of \textit{advanced telecommunication capabilities}. FCC considered these capabilities to require 4 Mbit/s download and 1 Mbit/s upload speeds in 2010, but revised the benchmark speeds to 25 Mbit/s download and 3 Mbit/s upload in 2015 \citep{fcc2015a}. In Europe, Finland was a worldwide pioneer in declaring affordable broadband access a basic right in 2010. Finland's Ministry of Transport and Communications set the threshold for the basic connection to 1 Mbit/s in 2010, revising it to 2 Mbit/s in 2016 \citep{eprs}. All these policy decisions have deep economic implications. Indeed, in most cases they imply the mobilization of universal service funds (USF) or subsidy schemes to meet the targets set in terms of availability of affordable Internet connections at a given speed.        
\end{itemize}

In addition to these consumer, policy and regulatory implications, broadband speeds may also be an important determinant of broadband impact on economic growth \citep{bohlin2012}. Higher broadband speeds have also been found to be causally linked to increases in the percentage of employees classified as creative class workers \citep{whitacre2014}. 

All these factors motivate the interest in producing accurate data on actual broadband speeds.     

\subsection{Research questions}
\vspace{1cm}
\setlength{\fboxsep}{1em}
\centerline{\fbox{\begin{minipage}{0.8\linewidth}
\begin{enumerate}
\item Can crowdsourcing Internet data be used to measure real broadband speeds?
\item Which information can be extracted from online speed measurements to characterize Internet users?   
\end{enumerate}
\end{minipage}}}
\vspace{1cm}
Given the relevance of broadband speeds for consumers, regulators and policy-makers, there is a growing demand for accurate measurements. 

Advertised speeds, as publicized by operators and ISPs, provide only an upper-limit to the actual broadband speeds. On the other hand, precise external hardware-based measurements, such as the ones commissioned to SamKnows by the regulatory agencies in the UK \citep{ofcom2017}, the United States \citep{fcc2015b} and the European Commission \citep{samknows2013} are costly. Therefore, they cannot be realistically scaled up to a wider set of countries.  

Software-based, crowdsourcing data on Internet speed measurements remains the only possible stable source of real broadband speed information for most countries. Moreover, the low cost of deployment of these measurement platforms makes it possible to envisage its adoption by any interested regulator/policy-making. 

Indeed, some regulators in developing countries, such as the Telecommunications Regulatory Commission of Sri Lanka, have already launched their own measurement portal (\autoref{fig:a2}). Moreover, there are private stakeholders, such as Ookla, recording these data at the global level.\footnote{For more information on Ookla's Speedtest, see \Cref{meth} and \href{http://beta.speedtest.net/about}{Speedtest website}.} 

\begin{figure}[H]
    \centering
        \includegraphics[width=\linewidth]{srilanka.pdf}
        \caption{Internet speed test platform,  Sri Lanka.}
        \caption*{\textbf{Source:} \href{http://www.trc.gov.lk/2014-05-12-13-25-54/internet-speed-test.html}{Telecommunications Regulatory Commission of Sri Lanka}.}
        \label{fig:a2}
\end{figure}   

In this context, the Internet Foundation in Sweden has been a forerunner of public-service broadband measurement platforms with its portal \textit{Bredbandskollen}, which collects data on Internet speeds since 2008 \citep{bredbandskollen}.   

This project takes advantage of the microdata made available by IIS to ITU on the broadband speed measurements from the \textit{Bredbandskollen} platform in the last six years (2011-2016). This large dataset is used as a testbed for determining to which extent this kind of crowdsourcing measurements can be used to provide robust insights into real broadband speeds, as well as information on Internet user behavior.  

\section{State of the art}
Broadband speeds are used by operators and ISPs as a means of characterizing their offers and segmenting their services according to different QoS. Advertised speeds are determined by each service provider according to their own internal methodology, which is not disclosed nor usually inspected by an independent party. However, it is understood that these speeds indicate the maximum or peak data rates that a customer may experience in the link between the customer location and the broadband provider \citep{bauer2010}.

More generally, advertised speeds are taken by customers to imply something meaningful about their experience when using a given broadband service. Since customers do not know in advance which Internet sites they will want to access, they demand universal connectivity from the ISPs and therefore expect advertised speeds to correspond to their end-to-end experience \citep[see the discussion on the MCI merger in][pp. 508-511]{economides2008}. 

For example, a customer contracting a 10 Mbit/s broadband plan with Movistar in Spain will expect to access information from The New York Times website at speeds close to 10 Mbit/s, even though the content of that website is hosted outside Movistar's network.      

Thus, the points from which to which the speed measurement is carried out matter for the consumer and explain part of the differences observed across different measurement platforms. \Cref{meth} reviews this and other technical factors having an impact on the state-of-the-art methodologies used to measure broadband speeds. \Cref{stats} presents how the problem of making statistical inference from crowdsourcing data on broadband measurements has been tackled in similar research efforts.   

\subsection{Measurement methodologies} \label{meth}

From a technical standpoint, the most accurate option for measuring real broadband speeds is to implement in-network measurements, such as real traffic monitoring (using network counters) or test-call routines. The technical issues concerning this type of measurements are well known. Indeed,  international standardization bodies, such as the European Telecommunications Standards Institute, have issued internationally agreed guidelines on how to perform this type of measurements \citep{etsi}.

In some countries, such as India and Spain, operators are required to self-report in-network measurements and the results are disclosed to the public on a regular basis \citep{setsi,zuhyle2015}.

However, in-network measurements can only be accomplished by those stakeholders having direct access to the network infrastructure and therefore rely on the self-reporting of the concerned network operators. Because of the difficulty of overseeing compliance and the lack of independence of this type of measurements, this approach is usually not considered a solution on its own, but rather a complement to other external measurements.

Therefore, external broadband speed measurements are the most common approach to gaining insights about real broadband speeds. There are a number of private stakeholders engaging in this kind of measurements and publishing data with a global coverage. These include, \textit{inter alia}, Ookla's Speedtest and Akamai speed reports \citep{bauer2010,lehr2013,bauer2016}. 

In addition, there exist several external broadband speed measurement platforms commissioned or operated by public entities and the academia. Some of them rely on hardware-based approaches, such as the SamKnows' Whitebox \citep{samknows2012,samknows2013} or the BISmark project's NoxBox \citep{sundaresan2012,sundaresan2014}. However, the majority of external measurements rely on purely software approaches. 

Examples of software-based approaches from public entities include the Regulatory Commission of Sri Lanka's Internet speed test platform \citep{zuhyle2015}, the Internet Foundation in Sweden's \textit{Bredbandskollen} and the Italian Authority for Communications Guarantees' \textit{Misurainternet} platform.\footnote{For more information, see the \href{https://www.misurainternet.it/}{Misurainternet website}}

These platforms collect several broadband performance metrics beyond download speed, such as upload speeds and latency (delay). Depending on the type of online activity performed, some parameters will be more important than others. For instance, latency is very relevant for real-time communications, such as VoIP. As a result of the diversity of activities performed online, a complete assessment of user quality of experience cannot be summarized into a single speed metric, but will require several complementary measurements \citep{samknows2013,zuhyle2015}. 

Moreover, there exist some application-specific broadband measurements, such as Netflix's Fast.com and Youtube's Speed numbers. These measurement platforms are optimized to reflect what is the actual speed experienced by a user transferring video files with the size and protocols common in Netflix and YouTube, respectively. However, these speed measurements may be a poor approximation of the actual speed experienced when browsing a website or sending a file \citep{bauer2010, bauer2016}.         

Based on a literature review of the state-of-the-art in broadband speed measurements, it can be concluded that external broadband speed measurement platforms depend on a few main design characteristics that affect their outcomes:

\begin{enumerate}[label=\textbf{\arabic* --}]
	\item \textbf{Measurement path:} from which point to which point of the network the measurement is made (\autoref{fig:a3}). Speeds advertised by operators refer to the maximum achievable over the access link (points 2-3 in \autoref{fig:a3}). In xDSL connections, the access link is not shared, as it is for coaxial cable, but xDSL speeds are more sensitive to the distance to the local exchange (path from 2 to 3). 
	
	From point 3 in \autoref{fig:a3}, there is contention for all wired technologies (fibre, coaxial, copper). That is, the transmission capacity is shared by many concurrent users and, depending on the network charge, this may lead to congestion and lower speeds. Even so, from points 2 to 4 the connection remains within the network of the ISP with whom the end user has contracted the service and therefore the speeds depend only on the network dimensioning and management of that ISP.
	
	End users' quality of experience is affected by the end-to-end path (i.e. points 1 to 5). This means that the speed experienced by a user also depends on the quality of the in-house connection (path from 1 to 2). For instance, the WiFi connection may not support high speeds or there might be several users making use of the same connection, therefore reducing the bandwidth available for the single user performing the test. Home network bottlenecks have been found to be very relevant in those settings in which the access is capable of providing speeds greater than 20 Mbit/s \citep{sundaresan2016}. 
	
\begin{figure}[H]
    \centering
        \includegraphics[width=\linewidth]{figure_network.png}
        \caption{Network diagram -- Key measurement points.}
        \caption*{\textbf{Source:} Author based on \cite{bauer2010}.}
        \label{fig:a3}
\end{figure}   
	
	In addition, end-to-end speeds may depend on other ISPs networks if the end user is accessing content hosted on the wider public Internet (off-net) instead of within the home ISP network (on-net). In the off-net case, the connection includes points 4 to 5 which are not under direct control of the home ISP. 
	
	However, it is the decision of the home ISP with whom to interconnect and under which conditions (e.g. traffic, peering). Therefore, under the assumption that customers demand universal connectivity from the ISP with whom they have the contract \citep[see][pp. 508-511]{economides2008}, this ISP could also be hold accountable for the speed delivered over the link from 4 to 5 in \autoref{fig:a3}. Indeed, ISP interconnection may have a substantial impact on consumer Internet performance and business relationships between ISPs are often at the root of this performance degradation, rather than technical issues \citep{m2014isp}.    

	\item \textbf{Active versus passive testing:} passive testing measures broadband speeds over end users' normal online activities, whereas active testing relies on standardized tests run independently of each end users' actual online activity. Active testing is usually preferred in benchmarking exercises because it facilitates the comparability between measurements \citep{zuhyle2015}. In passive testing, if two users are performing very different activities (e.g. heavy video streaming compared with light web browsing) the outcome of the speed measurements may be significantly different even if they have the same type of connection.

	\item \textbf{Voluntary versus automatic testing:} some broadband speed measurements are initiated by the end user (most software-based platforms, including Ookla and Bredbandskollen), whereas a few others are scheduled remotely and take place automatically (e.g. hardware-based platforms, such as SamKnows). Voluntary testing has several drawbacks, including the risk of selection bias. That is, end user's may run the test in a diagnostic fashion when they are experiencing network problems \citep{bauer2010}. Moreover, actual speeds are sensitive to congestion: they tend to decrease at peak hours or during traffic burst times \citep{sundaresan2012,zuhyle2015}. As a result, broadband speed results of voluntary tests may be biased because of the timing of the tests, which is out of control of the measurement platform and may not be randomly distributed. For instance, there could be more tests run during congestion periods because that is precisely when end users' perceive speed problems.          
	 
	\item \textbf{Data protocol configuration:} the data transmission protocol, usually TCP, may actually be the bottleneck if not correctly configured. This is more so in high-capacity networks (e.g. gigabit broadband networks), where a single TCP flow will most likely not be enough to measure the real capacity of the link. Therefore, multiple parallel TCP flows will be required to produce a reliable measurement in high-capacity networks \citep{bauer2010,bauer2016}.
	
	\item \textbf{Statistical aggregation:} results for individual tests are calculated using different methods. These include: (i) total bytes / total time, (ii) total bytes / total time after the ramp-up period, to exclude the initial warm-up period in which transmission is slower, and (iii) payload / bytes, so that only the actual information (and not the header aggregated by the transfer protocol) is considered. In addition, different aggregation methods are used for reporting aggregate statistics. For instance, SamKnows discards the top and bottom percentiles to control for outliers and averages the rest \citep{samknows2013}. Ookla removes the fastest 10 per cent and slowest 30 per cent slices of each measurement and averages the rest \citep{bauer2010}. \textit{Bredbandskollen} takes the higher between the average speed of the whole measurement period (10s) or the average of the last 8s.\footnote{See \textit{Bredbandskollen} methodological \href{https://ensupport.bredbandskollen.se/support/solutions/articles/1000228167-how-does-the-measurement-work-in-technical-terms-}{FAQ}.}   
	
	\item \textbf{Other design parameters}, such as test duration time, technology used in the application (e.g. HTML or Flash-based) and the file size used in download/upload measurements \citep{bauer2010,zuhyle2015}.             

\end{enumerate} 

\autoref{tab:t1} summarizes how this design parameters vary according to the broadband speed measurement platforms chosen in this project to benchmark the results obtained with \textit{Bredbandskollen}.

The main differences would be following: SamKnows ability to exclude the home network from the measurement and to automatically control the test timing, which will lead to more robust results; Akamai's passive testing, which may lead to comparing apples and oranges if their results are used to benchmark the outcome of active tests; the different statistical aggregation procedures which may potentially have a significant impact on the results.

Another relevant factor which needs to be considered is the different \textbf{sample sizes} of each broadband platform. For instance, concerning data from Sweden (i.e. the object of this project), \textit{Bredbandskollen} collected some 15 million observations per year for download speeds, Akamai 20 million and SamKnows 0.52 million. These differences illustrate the trade-off between volumes of data and precision of the measurement. Indeed, hardware-based measurements are costly and this severly limits the sample size even in developed countries.
\vspace{1cm}      
 
\begin{table}[H]
\bgroup
\def\arraystretch{1.5}%  1 is the default, change whatever you need
 \begin{tabular}{|p{3cm}|p{2.5cm}|p{1.6cm}|p{1.9cm}|p{2cm}|p{2.2cm}|}
  \hline 
   & \specialcell[c]{c}{\textbf{Measurement}\\ \textbf{path}} & \specialcell[c]{c}{\textbf{Active}\\\textbf{passive}} & \specialcell[c]{c}{\textbf{Voluntary}\\\textbf{automatic}} & \textbf{Data flows} & \specialcell[c]{c}{\textbf{Statistical}\\\textbf{aggregation}} \\ 
  \hline 
  \textbf{Akamai} & \specialcell[c]{c}{1-4 or 1-5\\depending on\\server load} & Passive & \specialcell[c]{c}{Real\\traffic} & Sequential & Unknown \\ 
  \hline 
  \textbf{Bredbandskollen} & \specialcell[c]{c}{1-4\\for most \\inland\\measurements} & Active & Voluntary & Parallel & \specialcell[c]{c}{Avg.first 2s\\or avg 10s} \\ 
  \hline 
  \textbf{Ookla} & \specialcell[c]{c}{1-4 or 1-5\\depending on\\user location} & Active & Voluntary & Parallel & \specialcell[c]{c}{Avg. after\\ramp-up\\Excl. top\\10\% and\\bottom 30\%\\slices} \\ 
  \hline 
  \textbf{SamKnows} & \specialcell[c]{c}{2-4 or 2-5\\depending on\\user location} & Active & Automatic & Parallel &  \specialcell[c]{c}{Avg. after\\ramp-up\\Excl. top\\and bottom\\percentiles}\\ 
  \hline 
  \end{tabular}
\egroup
  \caption{Main differences between selected Internet measurement platforms.}
  \caption*{\textbf{Source:} Author based on \cite{bauer2010, bauer2016,samknows2013} and \href{https://ensupport.bredbandskollen.se/support/solutions/articles/1000228167-how-does-the-measurement-work-in-technical-terms-}{Bredbandskollen}.}
        \label{tab:t1}  
\end{table}
  

\subsection{Statistical approach} \label{stats}
1st option you just describe the span of the sample is constructed, warn possible shortcomings and then just present the results.

\textbf{Add chart bubles Tiru}

\section{Data}
Bla bla
\subsection{Processing environment}
Bla bla
\section{Results}
Bla bla
\section{Conclusion}
Bla bla

%\nocite{*}

\bibliography{references}
\bibliographystyle{apalike}


\end{document}